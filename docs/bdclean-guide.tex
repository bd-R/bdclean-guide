\documentclass[]{book}
\usepackage{lmodern}
\usepackage{amssymb,amsmath}
\usepackage{ifxetex,ifluatex}
\usepackage{fixltx2e} % provides \textsubscript
\ifnum 0\ifxetex 1\fi\ifluatex 1\fi=0 % if pdftex
  \usepackage[T1]{fontenc}
  \usepackage[utf8]{inputenc}
\else % if luatex or xelatex
  \ifxetex
    \usepackage{mathspec}
  \else
    \usepackage{fontspec}
  \fi
  \defaultfontfeatures{Ligatures=TeX,Scale=MatchLowercase}
\fi
% use upquote if available, for straight quotes in verbatim environments
\IfFileExists{upquote.sty}{\usepackage{upquote}}{}
% use microtype if available
\IfFileExists{microtype.sty}{%
\usepackage{microtype}
\UseMicrotypeSet[protrusion]{basicmath} % disable protrusion for tt fonts
}{}
\usepackage[margin=1in]{geometry}
\usepackage{hyperref}
\hypersetup{unicode=true,
            pdftitle={bdclean User Guide},
            pdfauthor={Authors: Tomer Gueta and Thiloshon Nagarajah},
            pdfborder={0 0 0},
            breaklinks=true}
\urlstyle{same}  % don't use monospace font for urls
\usepackage{natbib}
\bibliographystyle{apalike}
\usepackage{color}
\usepackage{fancyvrb}
\newcommand{\VerbBar}{|}
\newcommand{\VERB}{\Verb[commandchars=\\\{\}]}
\DefineVerbatimEnvironment{Highlighting}{Verbatim}{commandchars=\\\{\}}
% Add ',fontsize=\small' for more characters per line
\usepackage{framed}
\definecolor{shadecolor}{RGB}{248,248,248}
\newenvironment{Shaded}{\begin{snugshade}}{\end{snugshade}}
\newcommand{\KeywordTok}[1]{\textcolor[rgb]{0.13,0.29,0.53}{\textbf{#1}}}
\newcommand{\DataTypeTok}[1]{\textcolor[rgb]{0.13,0.29,0.53}{#1}}
\newcommand{\DecValTok}[1]{\textcolor[rgb]{0.00,0.00,0.81}{#1}}
\newcommand{\BaseNTok}[1]{\textcolor[rgb]{0.00,0.00,0.81}{#1}}
\newcommand{\FloatTok}[1]{\textcolor[rgb]{0.00,0.00,0.81}{#1}}
\newcommand{\ConstantTok}[1]{\textcolor[rgb]{0.00,0.00,0.00}{#1}}
\newcommand{\CharTok}[1]{\textcolor[rgb]{0.31,0.60,0.02}{#1}}
\newcommand{\SpecialCharTok}[1]{\textcolor[rgb]{0.00,0.00,0.00}{#1}}
\newcommand{\StringTok}[1]{\textcolor[rgb]{0.31,0.60,0.02}{#1}}
\newcommand{\VerbatimStringTok}[1]{\textcolor[rgb]{0.31,0.60,0.02}{#1}}
\newcommand{\SpecialStringTok}[1]{\textcolor[rgb]{0.31,0.60,0.02}{#1}}
\newcommand{\ImportTok}[1]{#1}
\newcommand{\CommentTok}[1]{\textcolor[rgb]{0.56,0.35,0.01}{\textit{#1}}}
\newcommand{\DocumentationTok}[1]{\textcolor[rgb]{0.56,0.35,0.01}{\textbf{\textit{#1}}}}
\newcommand{\AnnotationTok}[1]{\textcolor[rgb]{0.56,0.35,0.01}{\textbf{\textit{#1}}}}
\newcommand{\CommentVarTok}[1]{\textcolor[rgb]{0.56,0.35,0.01}{\textbf{\textit{#1}}}}
\newcommand{\OtherTok}[1]{\textcolor[rgb]{0.56,0.35,0.01}{#1}}
\newcommand{\FunctionTok}[1]{\textcolor[rgb]{0.00,0.00,0.00}{#1}}
\newcommand{\VariableTok}[1]{\textcolor[rgb]{0.00,0.00,0.00}{#1}}
\newcommand{\ControlFlowTok}[1]{\textcolor[rgb]{0.13,0.29,0.53}{\textbf{#1}}}
\newcommand{\OperatorTok}[1]{\textcolor[rgb]{0.81,0.36,0.00}{\textbf{#1}}}
\newcommand{\BuiltInTok}[1]{#1}
\newcommand{\ExtensionTok}[1]{#1}
\newcommand{\PreprocessorTok}[1]{\textcolor[rgb]{0.56,0.35,0.01}{\textit{#1}}}
\newcommand{\AttributeTok}[1]{\textcolor[rgb]{0.77,0.63,0.00}{#1}}
\newcommand{\RegionMarkerTok}[1]{#1}
\newcommand{\InformationTok}[1]{\textcolor[rgb]{0.56,0.35,0.01}{\textbf{\textit{#1}}}}
\newcommand{\WarningTok}[1]{\textcolor[rgb]{0.56,0.35,0.01}{\textbf{\textit{#1}}}}
\newcommand{\AlertTok}[1]{\textcolor[rgb]{0.94,0.16,0.16}{#1}}
\newcommand{\ErrorTok}[1]{\textcolor[rgb]{0.64,0.00,0.00}{\textbf{#1}}}
\newcommand{\NormalTok}[1]{#1}
\usepackage{longtable,booktabs}
\usepackage{graphicx,grffile}
\makeatletter
\def\maxwidth{\ifdim\Gin@nat@width>\linewidth\linewidth\else\Gin@nat@width\fi}
\def\maxheight{\ifdim\Gin@nat@height>\textheight\textheight\else\Gin@nat@height\fi}
\makeatother
% Scale images if necessary, so that they will not overflow the page
% margins by default, and it is still possible to overwrite the defaults
% using explicit options in \includegraphics[width, height, ...]{}
\setkeys{Gin}{width=\maxwidth,height=\maxheight,keepaspectratio}
\IfFileExists{parskip.sty}{%
\usepackage{parskip}
}{% else
\setlength{\parindent}{0pt}
\setlength{\parskip}{6pt plus 2pt minus 1pt}
}
\setlength{\emergencystretch}{3em}  % prevent overfull lines
\providecommand{\tightlist}{%
  \setlength{\itemsep}{0pt}\setlength{\parskip}{0pt}}
\setcounter{secnumdepth}{5}
% Redefines (sub)paragraphs to behave more like sections
\ifx\paragraph\undefined\else
\let\oldparagraph\paragraph
\renewcommand{\paragraph}[1]{\oldparagraph{#1}\mbox{}}
\fi
\ifx\subparagraph\undefined\else
\let\oldsubparagraph\subparagraph
\renewcommand{\subparagraph}[1]{\oldsubparagraph{#1}\mbox{}}
\fi

%%% Use protect on footnotes to avoid problems with footnotes in titles
\let\rmarkdownfootnote\footnote%
\def\footnote{\protect\rmarkdownfootnote}

%%% Change title format to be more compact
\usepackage{titling}

% Create subtitle command for use in maketitle
\newcommand{\subtitle}[1]{
  \posttitle{
    \begin{center}\large#1\end{center}
    }
}

\setlength{\droptitle}{-2em}

  \title{\texttt{bdclean} User Guide}
    \pretitle{\vspace{\droptitle}\centering\huge}
  \posttitle{\par}
    \author{Authors: Tomer Gueta and Thiloshon Nagarajah}
    \preauthor{\centering\large\emph}
  \postauthor{\par}
      \predate{\centering\large\emph}
  \postdate{\par}
    \date{2018-09-05}

\usepackage{booktabs}

\begin{document}
\maketitle

{
\setcounter{tocdepth}{1}
\tableofcontents
}
\chapter*{Introduction}\label{introduction}
\addcontentsline{toc}{chapter}{Introduction}

\texttt{bdclean} is a user-friendly data cleaning Shiny app for the
inexperienced R user. It provides features to manage complete workflow
for biodiversity data cleaning, from uploading the data; gathering input
from the user, in order to adjust cleaning procedures; perform the
cleaning; and finally, generating various reports and several versions
of the data. \texttt{bdclean} is part of
\href{https://bd-r.github.io/The-bdverse/index.html}{The bdverse} -- a
collection of tools, that form a general framework for facilitating
biodiversity science in R.

\begin{figure}
\centering
\includegraphics{img/bdclean_bdverse.png}
\caption{bdclean in the bdverse}
\end{figure}

\subsubsection*{\texorpdfstring{\texttt{bdclean}
overview}{bdclean overview}}\label{bdclean-overview}
\addcontentsline{toc}{subsubsection}{\texttt{bdclean} overview}

\begin{figure}
\centering
\includegraphics{img/bdclean_overview.png}
\caption{bdclean overview}
\end{figure}

bdclean workflow is comprised of three distinct mechanisms, user input,
data cleaning and outputs. In most R packages this basic workflow
(i.e.~input; processing; output) operates via an R function. Functions
are fundamental building blocks of R, and usually focus on very specific
task. Users must understand and supply the function with its mandatory
arguments (e.g.~data in the specified format, setting of various
function variables). Thus, in order to create a specific workflow, users
must write an R script, which requires reasonable programing skills.
bdclean avoids all that by creating a user-friendly Shiny app with
questionnaire that collects the necessary user input.

\subsubsection*{App overview}\label{app-overview}
\addcontentsline{toc}{subsubsection}{App overview}

\subsubsection*{Fundings}\label{fundings}
\addcontentsline{toc}{subsubsection}{Fundings}

\begin{figure}
\centering
\includegraphics{img/ISF.png}
\caption{}
\end{figure}

\href{https://summerofcode.withgoogle.com/\%20target=\%22_blank\%22}{\includegraphics{img/GSoC.png}}

\href{https://github.com/rstats-gsoc/gsoc2018/wiki/bdclean\%3A-User-friendly-biodiversity-data-cleaning-pipeline\%20target=\%22_blank\%22}{See
the GSoC project idea page}

\chapter{\texorpdfstring{Installing
\texttt{bdclean}}{Installing bdclean}}\label{installing-bdclean}

\begin{center}\rule{0.5\linewidth}{\linethickness}\end{center}

\section{Development version from
GitHub}\label{development-version-from-github}

Windows users install
\href{https://cran.r-project.org/bin/windows/Rtools/}{Rtools} first.

\begin{Shaded}
\begin{Highlighting}[]
\KeywordTok{install.packages}\NormalTok{(}\StringTok{"devtools"}\NormalTok{)}
\NormalTok{devtools}\OperatorTok{::}\KeywordTok{install_github}\NormalTok{(}\StringTok{"bd-R/bdclean"}\NormalTok{)}
\CommentTok{# And also}
\NormalTok{devtools}\OperatorTok{::}\KeywordTok{install_github}\NormalTok{(}\StringTok{"bd-R/bdchecks"}\NormalTok{)}
\end{Highlighting}
\end{Shaded}

\textbf{To open the Shiny app, simply run:}

\begin{Shaded}
\begin{Highlighting}[]
\KeywordTok{run_bdclean}\NormalTok{()}
\end{Highlighting}
\end{Shaded}

\section{\texorpdfstring{{Very soon: a stable version from
CRAN}}{Very soon: a stable version from CRAN}}\label{very-soon-a-stable-version-from-cran}

\begin{Shaded}
\begin{Highlighting}[]
\KeywordTok{install.packages}\NormalTok{(}\StringTok{"bdDwC"}\NormalTok{)}
\end{Highlighting}
\end{Shaded}

\section{Possible problems \&
solutions}\label{possible-problems-solutions}

\textbf{{{[} TBA {]}}}

\subsection{???}\label{section}

TBA

\subsection{????}\label{section-1}

TBA

\chapter{Add data}\label{add-data}

\begin{center}\rule{0.5\linewidth}{\linethickness}\end{center}

\section{Load package}\label{load-package}

Load the \texttt{bdDwC} package

\begin{Shaded}
\begin{Highlighting}[]
    \KeywordTok{library}\NormalTok{(bdDwC)}
\end{Highlighting}
\end{Shaded}

\section{Darwinizing a dataset}\label{darwinizing-a-dataset}

\texttt{bdDwC} contains Indian Reptile dataset
\texttt{bdDwC:::dataReptiles}.

The function to Darwinize a dataset is\texttt{darwinizeNames} (replace
\texttt{bdDwC:::dataReptiles} with wanted dataset):

\begin{Shaded}
\begin{Highlighting}[]
\NormalTok{result <-}\StringTok{ }\KeywordTok{darwinizeNames}\NormalTok{(}\DataTypeTok{dataUser =}\NormalTok{ bdDwC}\OperatorTok{:::}\NormalTok{dataReptiles,}
                            \DataTypeTok{dataDWC   =}\NormalTok{ bdDwC}\OperatorTok{:::}\NormalTok{dataDarwinCloud}\OperatorTok{$}\NormalTok{data)}
\end{Highlighting}
\end{Shaded}

You can replace \texttt{bdDwC:::dataReptiles} with your dataset

Rename your dataset field names to Darwinized names using
\texttt{renameUserData}:

\begin{Shaded}
\begin{Highlighting}[]
\KeywordTok{renameUserData}\NormalTok{(bdDwC}\OperatorTok{:::}\NormalTok{dataReptiles, result)}
\end{Highlighting}
\end{Shaded}

\section{\texorpdfstring{Updating
\protect\hyperlink{the-darwin-cloud-dictionary}{the Darwin Cloud
dictionary}}{Updating the Darwin Cloud dictionary}}\label{updating-the-darwin-cloud-dictionary}

To get newest version of Darwin Cloud Data run:

\begin{Shaded}
\begin{Highlighting}[]
\KeywordTok{downloadCloudData}\NormalTok{()}
\end{Highlighting}
\end{Shaded}

which will download data from the remote repository and extract field
and standard names.

\chapter{Data cleaning configuration}\label{data-cleaning-configuration}

\begin{center}\rule{0.5\linewidth}{\linethickness}\end{center}

\section{Launching the app}\label{launching-the-app}

\begin{Shaded}
\begin{Highlighting}[]
\KeywordTok{library}\NormalTok{(bdDwC) }\CommentTok{# Uplaod package library}
\KeywordTok{runDwC}\NormalTok{() }\CommentTok{# Launch the app}
\end{Highlighting}
\end{Shaded}

\section{App overview}\label{app-overview-1}

\begin{figure}
\centering
\includegraphics{img/bdDwC_Getting_started.png}
\caption{bdDwC App Overview}
\end{figure}

In the first screen, you'll need to upload or download your biodiversity
data; choose dictionary and run the Darwinizer.

\section{Data upload}\label{data-upload}

\subsection{From a local file}\label{from-a-local-file}

A CSV file or a Darwin Core Archive (DwC-A) zip file can be uploaded.

\begin{figure}
\centering
\includegraphics{img/bdDwC_Up-local_file.png}
\caption{Data upload from a local file}
\end{figure}

\subsection{From an online database}\label{from-an-online-database}

Also, data can be retrieved directly from various online biodiversity
databases. You need only to:

\begin{itemize}
\tightlist
\item
  Select the database
\item
  Specify the desired scientific name.
\item
  Specify the number of records (upper limit of 50,000).
\item
  Check the box if records must have coordinates.
\item
  Wait for data to be downloaded.
\end{itemize}

\begin{figure}
\centering
\includegraphics{img/bdDwC_Up-database.png}
\caption{Data upload from online biodiversity databases}
\end{figure}

\section{Dictionaries}\label{dictionaries}

A dictionary is a key component when Darwinizing a dataset. It's
basically a lookup table that lists a possible variation of field name
and it corresponding DwC name.

\hypertarget{the-darwin-cloud-dictionary}{\subsection{The Darwin Cloud
dictionary}\label{the-darwin-cloud-dictionary}}

The Darwin Cloud dictionary \citep{DarwinCloud}, is a lookup table that
accumulates different variations in DwC field names from different
publishers. This valuable and critical dictionary was created and is
maintained by the Kurator project
(\url{http://kurator.acis.ufl.edu/kurator-web/}), which provides
workflow tools for data quality improvement of biodiversity data, via a
user-friendly web interface. The development of bdDwC was inspired by
\href{https://github.com/kurator-org/kurator-validation/wiki/CSV-File-Darwinizer\%20target=\%22_blank\%22}{Kurator's
own Darwinizer}.

\subsubsection*{Updating the Darwin
Cloud}\label{updating-the-darwin-cloud}
\addcontentsline{toc}{subsubsection}{Updating the Darwin Cloud}

It's recommended to update the Darwin Cloud file. This can be done
easily by clicking the \textbf{Update DC} button.

\begin{figure}
\centering
\includegraphics{img/bdDwC_update-DC.png}
\caption{Updating the Darwin Cloud}
\end{figure}

\subsection{Your own dictionary}\label{your-own-dictionary}

It's also possible to add your own dictionary by simply creating a CSV
file with two columns, one for the Field Names and one for the Standard
Names.

\begin{figure}
\centering
\includegraphics{img/bdDwC_personal_dictionary.png}
\caption{Uploading your own dictionary}
\end{figure}

\section{Darwinizing your dataset}\label{darwinizing-your-dataset}

Once a dataset is uploaded, the `Submit to Darwinizer' button is
activated, Clicking it will Darwinize the dataset.

\begin{figure}
\centering
\includegraphics{img/bdDwC_Submit.png}
\caption{Submit to Darwinizer button}
\end{figure}

\section{Darwinizer results}\label{darwinizer-results}

\subsection{Results page overwiew}\label{results-page-overwiew}

\begin{figure}
\centering
\includegraphics{img/bdDwC_Darwinizer_results.png}
\caption{Darwinizer results}
\end{figure}

Manually renaming field names can be done very easily, just choose the
two corresponding fields and click the Rename button.

\begin{figure}
\centering
\includegraphics{img/bdDwC_Manual_rename.png}
\caption{Manually renaming fields}
\end{figure}

Hovering over a DwC standard name will display its description.

\section{Download your Darwinized
data}\label{download-your-darwinized-data}

\section{Closing the app}\label{closing-the-app}

Just close the app browser tab, and the R session will be terminated. To
reopen it run in the R Console \texttt{runDwC()}.

\section{References}\label{references}

\chapter{Flag and clean the data}\label{flag-and-clean-the-data}

\begin{center}\rule{0.5\linewidth}{\linethickness}\end{center}

\section{Load package}\label{load-package-1}

Load the \texttt{bdDwC} package

\begin{Shaded}
\begin{Highlighting}[]
    \KeywordTok{library}\NormalTok{(bdDwC)}
\end{Highlighting}
\end{Shaded}

\section{Darwinizing a dataset}\label{darwinizing-a-dataset-1}

\texttt{bdDwC} contains Indian Reptile dataset
\texttt{bdDwC:::dataReptiles}.

The function to Darwinize a dataset is\texttt{darwinizeNames} (replace
\texttt{bdDwC:::dataReptiles} with wanted dataset):

\begin{Shaded}
\begin{Highlighting}[]
\NormalTok{result <-}\StringTok{ }\KeywordTok{darwinizeNames}\NormalTok{(}\DataTypeTok{dataUser =}\NormalTok{ bdDwC}\OperatorTok{:::}\NormalTok{dataReptiles,}
                            \DataTypeTok{dataDWC   =}\NormalTok{ bdDwC}\OperatorTok{:::}\NormalTok{dataDarwinCloud}\OperatorTok{$}\NormalTok{data)}
\end{Highlighting}
\end{Shaded}

You can replace \texttt{bdDwC:::dataReptiles} with your dataset

Rename your dataset field names to Darwinized names using
\texttt{renameUserData}:

\begin{Shaded}
\begin{Highlighting}[]
\KeywordTok{renameUserData}\NormalTok{(bdDwC}\OperatorTok{:::}\NormalTok{dataReptiles, result)}
\end{Highlighting}
\end{Shaded}

\section{\texorpdfstring{Updating
\protect\hyperlink{the-darwin-cloud-dictionary}{the Darwin Cloud
dictionary}}{Updating the Darwin Cloud dictionary}}\label{updating-the-darwin-cloud-dictionary-1}

To get newest version of Darwin Cloud Data run:

\begin{Shaded}
\begin{Highlighting}[]
\KeywordTok{downloadCloudData}\NormalTok{()}
\end{Highlighting}
\end{Shaded}

which will download data from the remote repository and extract field
and standard names.

\chapter{Artifacts and reports}\label{artifacts-and-reports}

\begin{center}\rule{0.5\linewidth}{\linethickness}\end{center}

\textbf{{{[} TBA {]}}}

\chapter{Getting your feedback}\label{getting-your-feedback}

\begin{center}\rule{0.5\linewidth}{\linethickness}\end{center}

Loading\ldots{}

\section{Report a bug}\label{report-a-bug}

Submit an issue at \url{https://github.com/bd-R/bdclean/issues}

\section{Contribute}\label{contribute}

Contribute: \url{https://github.com/bd-R/bdclean}

Join: \url{https://bd-r-group.slack.com}

\chapter{\texorpdfstring{\texttt{bdclean}
citation}{bdclean citation}}\label{bdclean-citation}

\begin{center}\rule{0.5\linewidth}{\linethickness}\end{center}

\begin{Shaded}
\begin{Highlighting}[]
\KeywordTok{citation}\NormalTok{(}\StringTok{"bdclean"}\NormalTok{)}
\end{Highlighting}
\end{Shaded}

\begin{verbatim}
## 
## To cite package 'bdclean' in publications use:
## 
##   Tomer Gueta, Thiloshon Nagarajah, Vijay Barve, Ashwin Agrawal,
##   Povilas Gibas and Yohay Carmel (2018). bdclean: A user-friendly
##   data cleaning app for the inexperienced R user. R package
##   version 0.1.900. https://github.com/bd-R/bdclean
## 
## A BibTeX entry for LaTeX users is
## 
##   @Manual{,
##     title = {bdclean: A user-friendly data cleaning app for the inexperienced R user},
##     author = {Tomer Gueta and Thiloshon Nagarajah and Vijay Barve and Ashwin Agrawal and Povilas Gibas and Yohay Carmel},
##     year = {2018},
##     note = {R package version 0.1.900},
##     url = {https://github.com/bd-R/bdclean},
##   }
\end{verbatim}

\chapter{Learn more about data
cleaning}\label{learn-more-about-data-cleaning}

\begin{center}\rule{0.5\linewidth}{\linethickness}\end{center}

\begin{itemize}
\item
  \textbf{\href{https://github.com/tdwg/dwc-qa\%20target=\%22_blank\%22}{The
  Darwin Core Questions \& Answers Site}}
\item
  \textbf{\href{https://github.com/tdwg/dwc-qa/wiki/Webinars\%20target=\%22_blank\%22}{Darwin
  Core Hour webinar series}}
\item
  \textbf{\href{https://github.com/tdwg/dwc-qa/wiki\%20target=\%22_blank\%22}{The
  Darwin Core Questions \& Answers wiki}}
\item
  \textbf{\href{https://www.gbif.org/darwin-core\%20target=\%22_blank\%22}{GBIF:
  What is Darwin Core, and why does it matter?}}
\item
  \textbf{\href{https://doi.org/10.1371/journal.pone.0029715\%20target=\%22_blank\%22}{Darwin
  Core: An Evolving Community-Developed Biodiversity Data Standard}
  \citep{DwC-paper} }
\end{itemize}

\subsubsection*{References}\label{references-1}
\addcontentsline{toc}{subsubsection}{References}

\bibliography{bib/book.bib,bib/DarwinCloud.bib,bib/DwC-paper.bib}


\end{document}
